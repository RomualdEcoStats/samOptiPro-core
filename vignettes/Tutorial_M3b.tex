% Options for packages loaded elsewhere
\PassOptionsToPackage{unicode}{hyperref}
\PassOptionsToPackage{hyphens}{url}
%
\documentclass[
  10pt,
  ignorenonframetext,
  aspectratio=169]{beamer}
\usepackage{pgfpages}
\setbeamertemplate{caption}[numbered]
\setbeamertemplate{caption label separator}{: }
\setbeamercolor{caption name}{fg=normal text.fg}
\beamertemplatenavigationsymbolsempty
% Prevent slide breaks in the middle of a paragraph
\widowpenalties 1 10000
\raggedbottom
\setbeamertemplate{part page}{
  \centering
  \begin{beamercolorbox}[sep=16pt,center]{part title}
    \usebeamerfont{part title}\insertpart\par
  \end{beamercolorbox}
}
\setbeamertemplate{section page}{
  \centering
  \begin{beamercolorbox}[sep=12pt,center]{section title}
    \usebeamerfont{section title}\insertsection\par
  \end{beamercolorbox}
}
\setbeamertemplate{subsection page}{
  \centering
  \begin{beamercolorbox}[sep=8pt,center]{subsection title}
    \usebeamerfont{subsection title}\insertsubsection\par
  \end{beamercolorbox}
}
\AtBeginPart{
  \frame{\partpage}
}
\AtBeginSection{
  \ifbibliography
  \else
    \frame{\sectionpage}
  \fi
}
\AtBeginSubsection{
  \frame{\subsectionpage}
}
\usepackage{amsmath,amssymb}
\usepackage{iftex}
\ifPDFTeX
  \usepackage[T1]{fontenc}
  \usepackage[utf8]{inputenc}
  \usepackage{textcomp} % provide euro and other symbols
\else % if luatex or xetex
  \usepackage{unicode-math} % this also loads fontspec
  \defaultfontfeatures{Scale=MatchLowercase}
  \defaultfontfeatures[\rmfamily]{Ligatures=TeX,Scale=1}
\fi
\usepackage{lmodern}
\usetheme[]{metropolis}
\usecolortheme{dove}
\usefonttheme{professionalfonts}
\ifPDFTeX\else
  % xetex/luatex font selection
\fi
% Use upquote if available, for straight quotes in verbatim environments
\IfFileExists{upquote.sty}{\usepackage{upquote}}{}
\IfFileExists{microtype.sty}{% use microtype if available
  \usepackage[]{microtype}
  \UseMicrotypeSet[protrusion]{basicmath} % disable protrusion for tt fonts
}{}
\makeatletter
\@ifundefined{KOMAClassName}{% if non-KOMA class
  \IfFileExists{parskip.sty}{%
    \usepackage{parskip}
  }{% else
    \setlength{\parindent}{0pt}
    \setlength{\parskip}{6pt plus 2pt minus 1pt}}
}{% if KOMA class
  \KOMAoptions{parskip=half}}
\makeatother
\usepackage{xcolor}
\newif\ifbibliography
\usepackage{listings}
\newcommand{\passthrough}[1]{#1}
\lstset{defaultdialect=[5.3]Lua}
\lstset{defaultdialect=[x86masm]Assembler}
\setlength{\emergencystretch}{3em} % prevent overfull lines
\providecommand{\tightlist}{%
  \setlength{\itemsep}{0pt}\setlength{\parskip}{0pt}}
\setcounter{secnumdepth}{-\maxdimen} % remove section numbering
% ===========================
% Thème & couleurs
% ===========================
\usetheme{Boadilla}
\usecolortheme{spruce}
\usefonttheme{default}             % Police beamer par défaut (LM)
\useoutertheme[subsection=true]{miniframes}

% On enlève les icônes de navigation + légendes numérotées
\setbeamertemplate{navigation symbols}{}
\setbeamertemplate{caption}[numbered]

% Palette couleurs (définitions sûres)
\usepackage{xcolor}
\definecolor{MSUgreen}{HTML}{18453B}   % vert soutenu lisible
\definecolor{HeadGray}{gray}{0.25}     % gris foncé pour bandeau
\definecolor{SubGray}{gray}{0.45}      % gris moyen pour sous-bandeau

% Couleurs du bandeau/entête miniframes
\setbeamercolor{section in head/foot}{fg=white,    bg=HeadGray}
\setbeamercolor{subsection in head/foot}{fg=white, bg=SubGray}

% (Facultatif) zones titre/auteur en pied de page si utilisées par ton template
\setbeamercolor{title in head/foot}{fg=black, bg=white!50!MSUgreen}
\setbeamercolor{author in head/foot}{fg=white!85!MSUgreen, bg=MSUgreen}

% ===========================
% Typo & microtypographie
% ===========================
\usepackage[T1]{fontenc}
\usepackage{lmodern}
\usepackage{microtype}               % meilleures césures/espaces

% Marges de texte un peu augmentées pour limiter les coupures
\setbeamersize{text margin left=0.9cm, text margin right=0.9cm}

% ===========================
% Images : jamais rognées
% ===========================
\usepackage{graphicx}
% Par défaut : occuper la largeur, limiter la hauteur, conserver le ratio
\setkeys{Gin}{width=\linewidth,height=0.80\textheight,keepaspectratio}
% (Optionnel) chemins d’images
% \graphicspath{{figs/}{plots/}{images/}}

% ===========================
% Code : listings + retour à la ligne
% ===========================
\usepackage{listings}
\lstset{
  basicstyle=\ttfamily\footnotesize,
  breaklines=true,
  breakatwhitespace=true,
  columns=fullflexible,
  keepspaces=true,
  showstringspaces=false,
  frame=single,
  framerule=0.3pt,
  tabsize=2,
  numbers=left,
  numberstyle=\tiny,
  numbersep=6pt
}
% (Optionnel) couleurs douces pour listings
\lstset{
  keywordstyle=\bfseries,
  commentstyle=\itshape,
  stringstyle={}
}
% (Optionnel) un style R simple
% \lstdefinelanguage{Rstat}{
%   language=R,
%   morekeywords={set.seed,library,readr,dplyr,ggplot2}
% }
% \lstset{language=Rstat}

% ===========================
% Titres longs & (cont.)
% ===========================
% Ajoute " (cont.) " automatiquement quand une frame est scindée par \framebreak
\setbeamertemplate{frametitle continuation}[from second][(cont.)]

% ===========================
% (Optionnel) Nettoyage de l’entête/pied Boadilla si besoin
% ===========================
% \setbeamertemplate{footline}[frame number]   % juste le numéro de slide
% \setbeamertemplate{headline}{}               % supprimer l’entête si tu préfères
\usepackage{bookmark}
\IfFileExists{xurl.sty}{\usepackage{xurl}}{} % add URL line breaks if available
\urlstyle{same}
\hypersetup{
  pdftitle={Optimizing Samplers in NIMBLE},
  pdfauthor={Romuald H.},
  hidelinks,
  pdfcreator={LaTeX via pandoc}}

\title{Optimizing Samplers in NIMBLE}
\author{Romuald H.}
\date{21 octobre 2025}

\begin{document}
\frame{\titlepage}

\begin{frame}[fragile]
\#Introduction

This tutorial demonstrates how to \textbf{diagnose differentiability}
and \textbf{optimize MCMC sampling strategies} in complex Bayesian
state--space models using the R package
\textbf{\passthrough{\lstinline!samOptiPro!}} (Hounyeme \emph{et al.},
2025).\\
We illustrate the workflow on a simple population dynamics model, using
\passthrough{\lstinline!nimble!} and \passthrough{\lstinline!nimbleHMC!}
as the computational back-end.

We will: - Build and simulate a \textbf{population growth model} with
latent process and log-normal observations. - Identify bottelenecks
(Algorithmic bottelenecks or and Time bottelelnecks) - Diagnose
\textbf{non-differentiable components} (to decide whether HMC/NUTS is
applicable). - Automatically \textbf{benchmark samplers} (RW, Slice,
AF\_slice, HMC, NUTS) using
\passthrough{\lstinline!test\_strategy\_block()!}. - Assess
\textbf{algorithmic (AE)} and \textbf{computational efficiency (CE)}.

All results (traceplots, diagnostics, and performance summaries) are
saved under \passthrough{\lstinline!outputs/!}.
\end{frame}

\begin{frame}[fragile]{Step 0- Load packages}
\phantomsection\label{step-0--load-packages}
\#Step 1. Simulated Data, Initial Values and monitors

\#Step 2. Model M3

\begin{lstlisting}
{
    for (t in 1:(n - 1)) {
        logit_theta[t] ~ dnorm(mean = 2, sd = 1)
        theta[t] <- ilogit(logit_theta[t])
    }
    N[1] ~ dlnorm(meanlog = 10, sdlog = 5)
    for (t in 1:(n - 1)) {
        N[t + 1] <- N[t] * theta[t]
    }
    Nobs[1] ~ dlnorm(meanlog = log(N[1]), sdlog = sd_obs[1])
    for (t in 3:n) {
        Nobs[t] ~ dlnorm(meanlog = log(N[t]), sdlog = sd_obs[t])
    }
}
\end{lstlisting}
\end{frame}

\begin{frame}{Step 3. Building and Compiling the Model}
\phantomsection\label{step-3.-building-and-compiling-the-model}
\end{frame}

\begin{frame}[fragile]{Step 4 . Diagnosing Differentiability and HMC
Eligibility}
\phantomsection\label{step-4-.-diagnosing-differentiability-and-hmc-eligibility}
\begin{lstlisting}

[MODEL STRUCTURE CHECK]
\end{lstlisting}

\begin{lstlisting}
- Stochastic nodes   : 8
\end{lstlisting}

\begin{lstlisting}
- Deterministic nodes: 21
\end{lstlisting}

\begin{lstlisting}

[STRUCTURE]
- #stochastic nodes   : 8
- #deterministic nodes: 21

[NON-DIFF INDICES]
- Non-diff functions detected : None
- Distributions found in code  : dlnorm, dnorm
- BUGS truncation ‘T(a,b)’ spotted : NO
- Bounded latent nodes (bounds on naturally unbounded support) : NO
- Naturally bounded latent nodes (e.g., dbeta/dunif/dtriangle)   : NO

[HMC/NUTS BLOCKERS]
- No explicit blockers detected at latent nodes.

- HMC globally possible ? Yes

[TEST DIFFERENTIABILITY / HMC]
===== Monitors =====
thin = 1: logit_theta, N
===== Samplers =====
RW sampler (8)
  - logit_theta[]  (7 elements)
  - N[]  (1 element)
thin = 1: logit_theta, N, theta
===== Monitors =====
thin = 1: logit_theta, N
===== Samplers =====
NUTS sampler (1)
  - logit_theta[1], logit_theta[2], logit_theta[3], logit_theta[4], logit_theta[5], logit_theta[6], logit_theta[7] 
===== Comments =====
\end{lstlisting}

\begin{lstlisting}
- HMC possible ? Yes
\end{lstlisting}
\end{frame}

\begin{frame}[fragile]{Step 5. Baseline MCMC, Bottelenecks and
Performance Assessment}
\phantomsection\label{step-5.-baseline-mcmc-bottelenecks-and-performance-assessment}
\begin{lstlisting}
===== Monitors =====
thin = 2: logit_theta, N, theta
thin2 = 1: logit_theta, N
===== Samplers =====
RW sampler (8)
  - logit_theta[]  (7 elements)
  - N[]  (1 element)
thin = 2: logit_theta, N, theta
thin2 = 1: logit_theta, N
thin = 2: logit_theta, N, theta
thin2 = 2: logit_theta, N
\end{lstlisting}

\begin{lstlisting}
|-------------|-------------|-------------|-------------|
|-------------------------------------------------------|
\end{lstlisting}

\begin{lstlisting}
|-------------|-------------|-------------|-------------|
|-------------------------------------------------------|
\end{lstlisting}

\begin{lstlisting}
|-------------|-------------|-------------|-------------|
|-------------------------------------------------------|
\end{lstlisting}

\begin{lstlisting}
[1] 21.226
\end{lstlisting}

\begin{lstlisting}
# A tibble: 1 x 11
  runtime_s n_chains n_iter n_params ESS_total ESS_per_s AE_mean AE_median
      <dbl>    <int>  <int>    <int>     <dbl>     <dbl>   <dbl>     <dbl>
1      21.2        3 495000       37 20730297.   976646.   0.377     0.324
# i 3 more variables: CE_mean <dbl>, CE_median <dbl>, prop_rhat_ok <dbl>
\end{lstlisting}

\begin{lstlisting}
    axis      family n_members    AE_med   CE_med    slowdown AE_rank CE_rank
1  joint       theta         7 0.2737351 19150.88 0.002460045       1       1
2  joint logit_theta        14 0.2839887 19868.24 0.002371223       2       2
3  joint           N        16 0.4283805 29970.08 0.001571969       3       3
4   algo       theta         7 0.2737351 19150.88 0.002460045       1       1
5   algo logit_theta        14 0.2839887 19868.24 0.002371223       2       2
6   algo           N        16 0.4283805 29970.08 0.001571969       3       3
7   comp       theta         7 0.2737351 19150.88 0.002460045       1       1
8   comp logit_theta        14 0.2839887 19868.24 0.002371223       2       2
9   comp           N        16 0.4283805 29970.08 0.001571969       3       3
10  slow       theta         7 0.2737351 19150.88 0.002460045       1       1
11  slow logit_theta        14 0.2839887 19868.24 0.002371223       2       2
12  slow           N        16 0.4283805 29970.08 0.001571969       3       3
   SLOW_rank JOINT JOINT_rank
1          1     3          1
2          2     6          2
3          3     9          3
4          1     3          1
5          2     6          2
6          3     9          3
7          1     3          1
8          2     6          2
9          3     9          3
10         1     3          1
11         2     6          2
12         3     9          3
\end{lstlisting}
\end{frame}

\begin{frame}[fragile]{Step 6-Adaptive Block Strategy ---
test\_strategy\_block()}
\phantomsection\label{step-6-adaptive-block-strategy-test_strategy_block}
\begin{lstlisting}
===== Monitors =====
thin = 2: logit_theta, N, theta
thin2 = 1: logit_theta, N
===== Samplers =====
RW sampler (8)
  - logit_theta[]  (7 elements)
  - N[]  (1 element)
thin = 2: logit_theta, N, theta
thin2 = 1: logit_theta, N
thin = 2: logit_theta, N, theta
thin2 = 2: logit_theta, N
===== Monitors =====
thin = 2: logit_theta, N, theta
thin2 = 1: logit_theta, N
===== Samplers =====
RW sampler (8)
  - logit_theta[]  (7 elements)
  - N[]  (1 element)
thin = 2: logit_theta, N, theta
thin2 = 1: logit_theta, N
thin = 2: logit_theta, N, theta
thin2 = 2: logit_theta, N
\end{lstlisting}

\begin{lstlisting}
|-------------|-------------|-------------|-------------|
|-------------------------------------------------------|
\end{lstlisting}

\begin{lstlisting}
|-------------|-------------|-------------|-------------|
|-------------------------------------------------------|
\end{lstlisting}

\begin{lstlisting}
|-------------|-------------|-------------|-------------|
|-------------------------------------------------------|
\end{lstlisting}

\begin{lstlisting}
[1] "familles ciblées: theta"
\end{lstlisting}

\begin{lstlisting}
[1] "HMC_full"
\end{lstlisting}

\begin{lstlisting}
              metric      value
1 ESS_per_sec_median  0.3207935
2          runtime_s 16.0590000
\end{lstlisting}

\begin{lstlisting}
              metric      value
1 ESS_per_sec_median  0.3207935
2          runtime_s 16.0590000
\end{lstlisting}
\end{frame}

\begin{frame}[fragile]{Step-7. Visualization and Diagnostics}
\phantomsection\label{step-7.-visualization-and-diagnostics}
\begin{lstlisting}[language=R]
diag_tbl <- compute_diag_from_mcmc(samples_ml, runtime_s = res_b$runtime_s)

plots_cv<-plot_convergence_checks(samples_ml,
  out_dir = "outputs/diagnostics",
  make_rhat_hist   = TRUE,
  make_rhat_ecdf   = TRUE,
  make_traces_rhat = FALSE,
  make_traces_ae   = FALSE
)

print(plots_cv$rhat_ecdf)
\end{lstlisting}

\begin{figure}
\includegraphics[width=0.9\linewidth]{Tutorial_M3b_files/figure-beamer/Visualization and Diagnostics-1} \end{figure}

\begin{lstlisting}[language=R]
print(plots_cv$rhat_hist)
\end{lstlisting}

\begin{figure}
\includegraphics[width=0.9\linewidth]{Tutorial_M3b_files/figure-beamer/Visualization and Diagnostics-2} \end{figure}

\begin{lstlisting}[language=R]
plots_bn <- plot_bottlenecks(
  diag_tbl,
  out_dir = "outputs/diagnostics",
  make_time_targets   = FALSE,
  make_esss_targets   = FALSE,
  make_esss_families  = TRUE,
  make_time_families  = TRUE,
  make_joint_targets  = FALSE,
  make_joint_families = FALSE,
  make_rhat_hist_targets    = FALSE,
  make_rhat_ecdf_targets    = TRUE,
  make_rhat_worst_targets   = FALSE,
  make_rhat_median_families = TRUE
)

print(plots_bn$rhat_ecdf_targets)
\end{lstlisting}

\begin{figure}
\includegraphics[width=0.9\linewidth]{Tutorial_M3b_files/figure-beamer/Visualization and Diagnostics-3} \end{figure}

\begin{lstlisting}[language=R]
print(plots_bn$rhat_median_families)
\end{lstlisting}

\begin{figure}
\includegraphics[width=0.9\linewidth]{Tutorial_M3b_files/figure-beamer/Visualization and Diagnostics-4} \end{figure}

\begin{lstlisting}[language=R]
print(plots_bn$time_families)
\end{lstlisting}

\begin{figure}
\includegraphics[width=0.9\linewidth]{Tutorial_M3b_files/figure-beamer/Visualization and Diagnostics-5} \end{figure}

\begin{lstlisting}[language=R]
print(plots_bn$esss_families)
\end{lstlisting}

\begin{figure}
\includegraphics[width=0.9\linewidth]{Tutorial_M3b_files/figure-beamer/Visualization and Diagnostics-6} \end{figure}

\begin{lstlisting}[language=R]
#The Empirical Cumulative Distribution Function (ECDF) displays the cumulative proportion of nodes whose Rhat is below a certain threshold.
#plot_bottlenecks() visualizes the worst nodes in terms of ESS/s (Algorithmic Efficiency) and runtime (Computational Efficiency).
#plot_convergence_checks() produces traceplots, autocorrelation, and R-hat diagnostics for the top problematic parameters
\end{lstlisting}

\#9. Conclusions

This workflow highlights how samOptiPro helps:

Detect non-differentiable nodes that prevent HMC/NUTS usage.

Automatically switch between gradient-based and non-gradient samplers.

Quantify algorithmic and computational efficiency for each parameter
family.

Provide transparent diagnostic plots and benchmark reports.

Even for a simple state--space model, the adaptive block design ensures
faster convergence and improved mixing without manual tuning.

\#10. References
\end{frame}

\end{document}
