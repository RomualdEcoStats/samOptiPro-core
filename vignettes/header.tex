% ===========================
% Thème & couleurs
% ===========================
\usetheme{Boadilla}
\usecolortheme{spruce}
\usefonttheme{default}             % Police beamer par défaut (LM)
\useoutertheme[subsection=true]{miniframes}

% On enlève les icônes de navigation + légendes numérotées
\setbeamertemplate{navigation symbols}{}
\setbeamertemplate{caption}[numbered]

% Palette couleurs (définitions sûres)
\usepackage{xcolor}
\definecolor{MSUgreen}{HTML}{18453B}   % vert soutenu lisible
\definecolor{HeadGray}{gray}{0.25}     % gris foncé pour bandeau
\definecolor{SubGray}{gray}{0.45}      % gris moyen pour sous-bandeau

% Couleurs du bandeau/entête miniframes
\setbeamercolor{section in head/foot}{fg=white,    bg=HeadGray}
\setbeamercolor{subsection in head/foot}{fg=white, bg=SubGray}

% (Facultatif) zones titre/auteur en pied de page si utilisées par ton template
\setbeamercolor{title in head/foot}{fg=black, bg=white!50!MSUgreen}
\setbeamercolor{author in head/foot}{fg=white!85!MSUgreen, bg=MSUgreen}

% ===========================
% Typo & microtypographie
% ===========================
\usepackage[T1]{fontenc}
\usepackage{lmodern}
\usepackage{microtype}               % meilleures césures/espaces

% Marges de texte un peu augmentées pour limiter les coupures
\setbeamersize{text margin left=0.9cm, text margin right=0.9cm}

% ===========================
% Images : jamais rognées
% ===========================
\usepackage{graphicx}
% Par défaut : occuper la largeur, limiter la hauteur, conserver le ratio
\setkeys{Gin}{width=\linewidth,height=0.80\textheight,keepaspectratio}
% (Optionnel) chemins d’images
% \graphicspath{{figs/}{plots/}{images/}}

% ===========================
% Code : listings + retour à la ligne
% ===========================
\usepackage{listings}
\lstset{
  basicstyle=\ttfamily\footnotesize,
  breaklines=true,
  breakatwhitespace=true,
  columns=fullflexible,
  keepspaces=true,
  showstringspaces=false,
  frame=single,
  framerule=0.3pt,
  tabsize=2,
  numbers=left,
  numberstyle=\tiny,
  numbersep=6pt
}
% (Optionnel) couleurs douces pour listings
\lstset{
  keywordstyle=\bfseries,
  commentstyle=\itshape,
  stringstyle={}
}
% (Optionnel) un style R simple
% \lstdefinelanguage{Rstat}{
%   language=R,
%   morekeywords={set.seed,library,readr,dplyr,ggplot2}
% }
% \lstset{language=Rstat}

% ===========================
% Titres longs & (cont.)
% ===========================
% Ajoute " (cont.) " automatiquement quand une frame est scindée par \framebreak
\setbeamertemplate{frametitle continuation}[from second][(cont.)]

% ===========================
% (Optionnel) Nettoyage de l’entête/pied Boadilla si besoin
% ===========================
% \setbeamertemplate{footline}[frame number]   % juste le numéro de slide
% \setbeamertemplate{headline}{}               % supprimer l’entête si tu préfères
